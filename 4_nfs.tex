\documentclass[11pt,a4paper,oneside]{article}
\usepackage[utf8]{inputenc}
\usepackage[T1]{fontenc}
\usepackage{lmodern}
\usepackage[french]{babel}
\pagestyle{headings}
\author{Bruno Parmentier \and Olivier Marcotte}
\title{Gestion et administration des réseaux - Linux \\[1cm] \emph{Mise en place
d'un partage NFS}}
\date{27 mars 2014}
\begin{document}

\begin{titlepage}
    \maketitle
    \thispagestyle{empty}
\end{titlepage}

\section{Objectif}
Mise en place d'un partage de dossier en NFS sur \emph{darkstar} (la
passerelle).

Les deux machines \emph{local1} et \emph{local2} auront accès au dossier partagé
avec différentes permissions et options.

\section{Mode opératoire}

\subsection{Configuration du serveur de fichier NFS}
Nous créons d'abord le dossier \verb#/tmp/public# qui sera le dossier partagé.

Dans le fichier \verb#/etc/exports#~:
\begin{verbatim}
/tmp/public 172.16.24.2(rw,async) 172.16.24.3(rw,no_root_squash)
\end{verbatim}

Pour partager le dossier pour un (sous-)réseau complet, on peut modifier la
ligne comme suit~:
\begin{verbatim}
/tmp/public 172.16.24.0/24(rw,sync)
\end{verbatim}

Voici un petit résumé des options testées~:
\begin{itemize}
    \item \verb#rw#~: le dossier est en lecture/écriture~;
    \item \verb#ro#~: le dossier est en lecture uniquement~;
    \item \verb#sync#~: mode par défaut, les fichiers sont synchronisés en
        continu~;
    \item \verb#async#~: permet d'augmenter les performance, au risque d'avoir
        des données corrompues en cas de crash du serveur~;
    \item \verb#root_squash#~: option par défaut, l'uid et le gid d'un fichier
        créé par un client root sont mappés sur 65534 (\emph{nobody})~;
    \item \verb#no_root_squash#~: l'uid et le gid d'un fichier créé par un
        client root restent sur 0 (root).
\end{itemize}

Pour lancer le démon du serveur NFS, exécuter la commande
\verb#/etc/rc.d/rc.nfsd start#.

Pour recharger la table des systèmes de fichiers NFS exportés, sans couper la
connexion (après une modification du fichier \verb#/etc/exports# par exemple),
exécuter la commande suivante~:
\begin{verbatim}
exportfs -rf
\end{verbatim}

Pour lancer le démon nfsd au démarrage de l'ordinateur, rendre exécutable le
fichier \verb#/etc/rc.d/rc.nfsd#.

On peut également configurer les permissions d'accès des clients au serveur via
les fichiers \verb#/etc/hosts.allow# et \verb#/etc/hosts.deny#.

Exemple pour \verb#/etc/hosts.deny# (interdit l'accès depuis la machine
\verb#172.16.24.2#)~:
\begin{verbatim}
# /etc/hosts.deny

lockd: 172.16.24.2
rquotad: 172.16.24.2
mountd: 172.16.24.2
statd: 172.16.24.2
\end{verbatim}

\subsection{Configuration des clients}
Montons le dossier partagé par \emph{darkstar} sur un dossier local,
\verb#/mnt/tmp# par exemple. Pour le monter directement, activer d'abord portmap
et statd avec la commande \verb#/etc/rc.d/rc.rpc start#.

Monter ensuite le système de fichier NFS~:
\begin{verbatim}
mount 172.16.24.1:/tmp/public /mnt/tmp
\end{verbatim}

Afin d'activer le montage du dossier au démarrage du client, rendre exécutable
le fichier \verb#/etc/rc.d/rc.rpc# et ajouter la ligne suivante dans
\verb#/etc/fstab#~:
\begin{verbatim}
172.16.24.1:/tmp/public    /mnt/tmp    nfs     rw      0   0
\end{verbatim}

\section{Conclusion}
Le partage d'un dossier a bien été effectué. Nous avons pu vérifier et comparer
les options spécifiques à chaque client indiquées dans le fichier
\verb#/etc/exports# du serveur.

\end{document}
