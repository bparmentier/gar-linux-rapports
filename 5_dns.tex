\documentclass[11pt,a4paper,oneside]{article}
\usepackage[utf8]{inputenc}
\usepackage[T1]{fontenc}
\usepackage{lmodern}
\usepackage{hyperref}
\usepackage{color}
\definecolor{gray}{rgb}{0.9,0.9,0.9}
\usepackage{listings}
\lstset{basicstyle=\ttfamily,
    backgroundcolor=\color{gray},
    captionpos=b,
    keepspaces=true,
    columns=flexible}
\lstset{literate=
    {á}{{\'a}}1 {é}{{\'e}}1 {í}{{\'i}}1 {ó}{{\'o}}1 {ú}{{\'u}}1
    {Á}{{\'A}}1 {É}{{\'E}}1 {Í}{{\'I}}1 {Ó}{{\'O}}1 {Ú}{{\'U}}1
    {à}{{\`a}}1 {è}{{\`e}}1 {ì}{{\`i}}1 {ò}{{\`o}}1 {ù}{{\`u}}1
    {À}{{\`A}}1 {È}{{\'E}}1 {Ì}{{\`I}}1 {Ò}{{\`O}}1 {Ù}{{\`U}}1
    {ä}{{\"a}}1 {ë}{{\"e}}1 {ï}{{\"i}}1 {ö}{{\"o}}1 {ü}{{\"u}}1
    {Ä}{{\"A}}1 {Ë}{{\"E}}1 {Ï}{{\"I}}1 {Ö}{{\"O}}1 {Ü}{{\"U}}1
    {â}{{\^a}}1 {ê}{{\^e}}1 {î}{{\^i}}1 {ô}{{\^o}}1 {û}{{\^u}}1
    {Â}{{\^A}}1 {Ê}{{\^E}}1 {Î}{{\^I}}1 {Ô}{{\^O}}1 {Û}{{\^U}}1
    {œ}{{\oe}}1 {Œ}{{\OE}}1 {æ}{{\ae}}1 {Æ}{{\AE}}1 {ß}{{\ss}}1
    {ç}{{\c c}}1 {Ç}{{\c C}}1 {ø}{{\o}}1 {å}{{\r a}}1 {Å}{{\r A}}1
    {€}{{\EUR}}1 {£}{{\pounds}}1
}
\newcommand{\inlinecode}{\lstinline[breaklines=true]}
\usepackage[french]{babel}
\usepackage{caption}
\pagestyle{headings}
\author{Bruno Parmentier \and Olivier Marcotte}
\title{Gestion et administration des réseaux - Linux \\[1cm] \emph{Configuration
de serveurs DNS}}
\date{24 avril 2014}
\begin{document}

\begin{titlepage}
    \maketitle
    \thispagestyle{empty}
\end{titlepage}

\section{Introduction}
Le DNS (\emph{Domain Name System}) permet de traduire un nom de domaine en
adresse IP, et inversement. Il est organisé de façon hiérarchique. Un domaine
tel que \url{www.opennic.org.} se lit de droite à gauche~:
\begin{itemize}
    \item la racine (\textbf{.}) est gérée par les 13 serveurs roots~;
    \item le TLD, ou \emph{top-level domain} (\textbf{org}) est délégué à une
        organisation par pays (ccTLD~: .be, .fr, .sx, etc.) ou générique (gTLD~:
        .com, .org, .net, etc.)~;
    \item le SLD, ou \emph{second-level domain} (\textbf{opennic}) est
        enregistrée auprès de l'organisation gérant le TLD correspondant~;
    \item les sous-domaines suivants (\textbf{www}).
\end{itemize}

Avant l'adoption des serveurs DNS, la résolution des noms se faisait sur chaque
machine dans le fichier \inlinecode{hosts}. Avec DNS, cette résolution n'est
plus liée à un fichier, elle est dynamique.

\section{Objectif}
Mise en place d'un serveur DNS maître sur la gateway et d'un esclave sur la
machine de la DMZ.

Nous utiliserons le logiciel \textbf{BIND} comme serveur DNS.

\section{Mode opératoire}

Sur les deux machines, démarrer bind avec la commande
\inlinecode{/etc/rc.d/rc.bind start}. Rendre le fichier correspondant exécutable
avec la commande \inlinecode{chmod +x /etc/rc.d/rc.bind} afin de lancer le
serveur au démarrage.

\subsection{Configuration du serveur maître}

La déclaration des zones se fait dans le fichier \inlinecode{/etc/named.conf}.
Nous devons créer deux zones, \textbf{dmz4} et \textbf{local4}, respectivement
pour nos sous-réseaux \inlinecode{172.16.14.0/24} et
\inlinecode{172.16.24.0/24}.

Ajouter les lignes suivantes dans \inlinecode{/etc/named.conf}~:
\begin{lstlisting}[caption=\inlinecode{/etc/named.conf}]
zone "local4" IN {
	type master;
	file "caching-example/local4.zone";
};

zone "24.16.172.in-addr.arpa" IN {
	type master;
	file "caching-example/local4.rev";
};

zone "dmz4" IN {
	type master;
	file "caching-example/dmz4.zone";
};

zone "14.16.172.in-addr.arpa" IN {
	type master;
	file "caching-example/dmz4.rev";
};
\end{lstlisting}

Ensuite, créer les fichiers de zones comme suit dans le répertoire
\inlinecode{/var/named/caching-example/} (les enregistrements DNS peuvent être
de type IN, A, AAAA, NS, PTR, MX, etc.)~:
\lstinputlisting[caption=\inlinecode{/var/named/caching-example/dmz4.zone}]{resources/dmz4.zone}
\lstinputlisting[caption=\inlinecode{/var/named/caching-example/dmz4.rev}]{resources/dmz4.rev}
\lstinputlisting[caption=\inlinecode{/var/named/caching-example/local4.zone}]{resources/local4.zone}
\lstinputlisting[caption=\inlinecode{/var/named/caching-example/local4.rev}]{resources/local4.rev}

Indiquer enfin dans \inlinecode{/etc/resolv.conf} l'adresse du serveur DNS à
utiliser, la boucle local~:
\begin{lstlisting}[caption=\inlinecode{/etc/resolv.conf}]
nameserver 127.0.0.1
\end{lstlisting}

\subsection{Configuration du serveur esclave}

Le serveur esclave doit uniquement connaître l'adresse de son maître. Ceci se
fait dans son fichier \inlinecode{/etc/named.conf} en ajoutant la ligne
\inlinecode{masters} et en définissant le \inlinecode{type} comme
\inlinecode{slave} dans chaque zone~:
\begin{lstlisting}[caption=\inlinecode{/etc/named.conf}]
zone "local4" IN {
    type slave;
    masters 172.16.24.1;
    file "caching-example/local4.zone";
};

zone "24.16.172.in-addr.arpa" IN {
    type slave;
    masters 172.16.24.1;
    file "caching-example/local4.rev";
};

zone "dmz4" IN {
    type slave;
    masters 172.16.14.1;
    file "caching-example/dmz4.zone";
};

zone "14.16.172.in-addr.arpa" IN {
    type slave;
    masters 172.16.14.1;
    file "caching-example/dmz4.rev";
};
\end{lstlisting}

\subsection{Vérification du fonctionnement des serveurs}

Remplacer dans le fichier \inlinecode{/etc/dhcpd.conf} sur la gateway (serveur
DHCP) l'adresse du serveur DNS actuel par celle de notre nouveau serveur DNS
esclave (\inlinecode{172.16.14.2}). Ajouter les lignes suivantes~:
\begin{lstlisting}[caption=\inlinecode{/etc/dhcpd.conf}]
option domain-search "dmz4", "local4";
option domain-name-servers 172.16.14.2;
\end{lstlisting}

Relancer \textbf{dhcpd} sur la gateway, puis \textbf{dhcpcd} sur les clients.
Dorénavant, les requêtes DNS seront traitées par notre serveur esclave.

Sur le serveur maître, modifier les fichiers de zones et tester différentes
valeurs de rafraichissement. Ne pas oublier d'incrémenter la valeur du numéro de
série (par convention, sous la forme AAAAMMJJSS, où AAAA est l'année, MM est le
mois, JJ est le jour et SS est un numéro de séquence).

Vérifier que le contenu du répertoire \inlinecode{/var/named/caching-example/}
sur le serveur esclave a bien été récupéré depuis le serveur maître.

\subsection{Divers}
\begin{itemize}
    \item les commandes \textbf{dig} et \textbf{nslookup} peuvent être utiles
        pour questionner les serveurs DNS (vérifier notamment le temps de
        réponse après deux requêtes vers un même site pour déterminer si le
        serveur DNS fait du caching)~;
    \item \textbf{tcpdump} permet de sniffer le traffic sur une interface
        réseau. L'on peut ainsi vérifier le passage de paquets DNS NOTIFY depuis
        le serveur maître vers l'exclave~;
    \item \textbf{named-checkconf} permet de vérifier la syntaxe de
        \inlinecode{/etc/named.conf}, tandis que \textbf{named-checkzone}
        vérifie l'intégrité et la syntaxe d'un fichier de zone~;
\end{itemize}

\section{Conclusion}

Nous avons mis en place deux serveurs DNS~: un maître et un esclave. Le maître
s'occupe de définir et gérer les zones et l'esclave, quant à lui, traite les
requêtes des clients et fait du caching. L'esclave se met correctement à jour lorsqu'une
modification est effectuée sur le maître.

\end{document}
