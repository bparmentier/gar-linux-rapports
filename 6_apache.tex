\documentclass[11pt,a4paper,oneside]{article}
\usepackage[utf8]{inputenc}
\usepackage[T1]{fontenc}
\usepackage{lmodern}
\usepackage{color}
\definecolor{gray}{rgb}{0.9,0.9,0.9}
\usepackage{listings}
\lstset{basicstyle=\ttfamily,
    backgroundcolor=\color{gray},
    keepspaces=true,
    columns=flexible}
\lstset{literate=
    {á}{{\'a}}1 {é}{{\'e}}1 {í}{{\'i}}1 {ó}{{\'o}}1 {ú}{{\'u}}1
    {Á}{{\'A}}1 {É}{{\'E}}1 {Í}{{\'I}}1 {Ó}{{\'O}}1 {Ú}{{\'U}}1
    {à}{{\`a}}1 {è}{{\`e}}1 {ì}{{\`i}}1 {ò}{{\`o}}1 {ù}{{\`u}}1
    {À}{{\`A}}1 {È}{{\'E}}1 {Ì}{{\`I}}1 {Ò}{{\`O}}1 {Ù}{{\`U}}1
    {ä}{{\"a}}1 {ë}{{\"e}}1 {ï}{{\"i}}1 {ö}{{\"o}}1 {ü}{{\"u}}1
    {Ä}{{\"A}}1 {Ë}{{\"E}}1 {Ï}{{\"I}}1 {Ö}{{\"O}}1 {Ü}{{\"U}}1
    {â}{{\^a}}1 {ê}{{\^e}}1 {î}{{\^i}}1 {ô}{{\^o}}1 {û}{{\^u}}1
    {Â}{{\^A}}1 {Ê}{{\^E}}1 {Î}{{\^I}}1 {Ô}{{\^O}}1 {Û}{{\^U}}1
    {œ}{{\oe}}1 {Œ}{{\OE}}1 {æ}{{\ae}}1 {Æ}{{\AE}}1 {ß}{{\ss}}1
    {ç}{{\c c}}1 {Ç}{{\c C}}1 {ø}{{\o}}1 {å}{{\r a}}1 {Å}{{\r A}}1
    {€}{{\EUR}}1 {£}{{\pounds}}1
}
\newcommand{\inlinecode}{\lstinline[breaklines=true]}
\usepackage[french]{babel}
\pagestyle{headings}
\author{Bruno Parmentier \and Olivier Marcotte}
\title{Gestion et administration des réseaux - Linux \\[1cm] \emph{Configuration
d'un serveur Apache}}
\date{8 mai 2014}
\begin{document}

\begin{titlepage}
    \maketitle
    \thispagestyle{empty}
\end{titlepage}

\section{Objectif}
Mise en place d'un serveur web Apache sur la machine de la DMZ.

\section{Mode opératoire}

\subsection{Démarrage d'Apache}

La configuration d'Apache se fait dans le fichier \inlinecode{/etc/httpd/httpd.conf}
ainsi que dans les fichiers contenus dans le répertoire
\inlinecode{/etc/httpd/httpd/extra/}.
Nous démarrons le démon avec la commande \inlinecode{/etc/rc.d/rc.httpd start}. Pour
lancer le serveur au démarrage de la machine, rendre exécutable le script avec
la commande \inlinecode{chmod +x /etc/rc.d/rc.httpd}.

\subsection{Vérification du nombre de démons}

Nous pouvons vérifier le nombre de démons lancés avec la commande \inlinecode{htop} ou
\inlinecode{ps aux | grep httpd}.

\subsection{Augmentation du nombre de démons}

Le nombre de démons qu'Apache est autorisé à lancer est à configurer dans le
fichier \inlinecode{/etc/httpd/extra/httpd-mpm.conf}.

\subsection{Modification de la page d'accueil}

La page d'accueil par défaut (\inlinecode{index.html}) se trouve dans le dossier
\inlinecode{/srv/httpd/htdocs/}. Nous modifions cette page et vérifions que celle-ci
apparaît bien lorsque l'on entre l'adresse IP du serveur (\inlinecode{172.16.14.2})
dans un navigateur d'une autre machine du réseau.

\subsection{Configuration de sites par utilisateur}

La configuration de sites par utilisateurs permet à chaque utilisateur ayant un
compte système d'avoir un espace web personnel, mais en utilisant chacun le même
nom de domaine. La configuration du module se fait dans le fichier
\inlinecode{/etc/httpd/extra/httpd-userdir.conf}. Décommenter également les lignes
suivantes dans \inlinecode{/etc/httpd/httpd.conf}~:

\begin{lstlisting}
LoadModule userdir_module lib/httpd/modules/mod_userdir.so
\end{lstlisting}

\begin{lstlisting}
Include /etc/httpd/extra/httpd-userdir.conf
\end{lstlisting}

Nous créons un utilisateur \textbf{marpar} avec la commande suivante~:
\begin{lstlisting}
useradd -m marpar
\end{lstlisting}
\inlinecode{-m} permettant de créer un répertoire personnel dans
\inlinecode{/home}.

Les fichiers du site sont à placer dans le dossier
\inlinecode{~/public_html/} et la page d'accueil de chaque site aura une adresse de la
forme \inlinecode{http://www.domain.tld/~username}.

Par exemple, le fichier \inlinecode{/home/marpar/public_html/index.html} sera dans
notre cas accessible à l'adresse \inlinecode{http://dmz-server.dmz4/~marpar} ou
\inlinecode{http://172.16.14.2/~marpar}.

\subsection{Configuration de site par adresse IP}

Une deuxième solution pour mettre en place plusieurs sites internet sur une même
machine consiste à attribuer une adresse IP à chacun des sites. Le nombre de
cartes réseaux étant limité sur la machine, il est possible de créer des
interfaces virtuelles. Ceci est réalisé avec la commande suivante~:
\begin{lstlisting}
ifconfig eth0:1 172.16.14.20/24
\end{lstlisting}

Afin d'activer le support des vhosts dans Apache, décommenter la ligne suivante
dans \inlinecode{/etc/httpd/httpd.conf}~:
\begin{lstlisting}
Include /etc/httpd/extra/httpd-vhosts.conf
\end{lstlisting}

Modifier ensuite le fichier \inlinecode{/etc/httpd/extra/httpd-vhosts.conf} comme
suit~:
\begin{lstlisting}
<VirtualHost 172.16.14.20:80>
  ServerAdmin   webmaster@marpar.dmz4
  DocumentRoot  "/srv/httpd/htdocs/virtualhost"
  ServerName    marpar.dmz4
  ErrorLog      "/var/log/httpd/marpar.dmz4-error_log"
  CustomLog     "/var/log/httpd/marpar.dmz4-access_log" common
</VirtualHost>
\end{lstlisting}

Ajouter le sous-domaine \textbf{marpar.dmz4} au serveur DNS (ne pas oublier de
changer le numéro de série du fichier de zone)~:
\begin{lstlisting}
marpar.dmz4.    1D  IN  A   172.16.14.20
\end{lstlisting}

\subsection{Configuration de site par nom}

La troisième solution permet de configurer plusieurs sites en fonction de leur
nom de domaine, le tout sur une même interface réseau. Nous créons deux sites
\inlinecode{xxx.dmz4} et \inlinecode{yyy.dmz4} dans
\inlinecode{/etc/httpd/extra/httpd-vhosts.conf}~:
\begin{lstlisting}
<VirtualHost 172.16.14.2:80>
  ServerAdmin   webmaster@xxx.dmz4
  DocumentRoot  "/srv/httpd/htdocs/xxx.dmz4"
  ServerName    xxx.dmz4
  ErrorLog      "/var/log/httpd/xxx.dmz4-error_log"
  CustomLog     "/var/log/httpd/xxx.dmz4-access_log" common
</VirtualHost>

<VirtualHost 172.16.14.2:80>
  ServerAdmin   webmaster@yyy.dmz4
  DocumentRoot  "/srv/httpd/htdocs/yyy.dmz4"
  ServerName    yyy.dmz4
  ErrorLog      "/var/log/httpd/yyy.dmz4-error_log"
  CustomLog     "/var/log/httpd/yyy.dmz4-access_log" common
</VirtualHost>
\end{lstlisting}

Ajouter les sous-domaines \textbf{xxx.dmz4} et \textbf{yyy.dmz4} au serveur
DNS~:
\begin{lstlisting}
xxx.dmz4.       1D  IN  A   172.16.14.2
yyy.dmz4.       1D  IN  A   172.16.14.2
\end{lstlisting}

\subsection{Suppléments}
\subsubsection{Page d'erreur}
Ajoutez la ligne suivante dans le \inlinecode{VirtualHost} désiré ou dans un fichier
\inlinecode{.htaccess} situé à la racine du site ciblé~:
\begin{lstlisting}
ErrorDocument       404 /404.html
\end{lstlisting}

\subsubsection{Changement du port d'écoute}
Le port d'écoute peut être changé pour tous les sites dans le fichier
\inlinecode{/etc/httpd/httpd.conf} avec l'option \inlinecode{Listen XX}.

\subsubsection{Sécurisation d'un site par mot de passe}
Créer un fichier \inlinecode{.htpasswd} avec les noms d'utilisateurs et mots de passe
des personnes autorisées à y accéder. Ce fichier doit être créé avec la commande
\inlinecode{htpasswd}~:
\begin{lstlisting}
htpasswd -B -c /etc/http.passwd <username>
\end{lstlisting}
\inlinecode{-B} permettant d'utiliser l'algorithme bcrypt pour le chiffrage du mot de
passe et \inlinecode{-c} définissant le nom du fichier cible.

Enfin, ajouter à \inlinecode{.htaccess} les lignes suivantes~:
\begin{lstlisting}
AuthType            Basic
AuthName            "Zone sécurisée"
AuthUserFile        /etc/http.passwd
Require             user <username>
\end{lstlisting}

\section{Conclusion}

Tous les sites créés sont fonctionnels et sont accessibles depuis les autres
machines de notre réseau (\inlinecode{dmz4} et \inlinecode{local4}).

\end{document}
