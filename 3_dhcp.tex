\documentclass[11pt,a4paper,oneside]{article}
\usepackage[utf8]{inputenc}
\usepackage[T1]{fontenc}
\usepackage{lmodern}
\usepackage[french]{babel}
\pagestyle{headings}
\author{Bruno Parmentier \and Olivier Marcotte}
\title{Gestion et administration des réseaux - Linux \\[1cm] \emph{Mise en place
d'un serveur DHCP}}
\date{27 février 2014}
\begin{document}

\begin{titlepage}
    \maketitle
    \thispagestyle{empty}
\end{titlepage}

\section{Objectif}
Configuration du réseau avec adressage dynamique.

La machine \emph{darkstar} (la passerelle) fera office de serveur DHCP. Les
trois autres machines du réseau obtiendront des adresses IP fixes en DHCP. Toute
autre machine voulant se connecter au réseau obtiendra une adresse IP dynamique
sur le réseau local dans le \emph{range} \verb#172.16.24.4-254#.

\section{Mode opératoire}

\subsection{Configuration du serveur DHCP}
Dans le fichier \verb#/etc/dhcpd.conf#~:
\begin{verbatim}
# domaine
option domain-name "marpar.org";
option domain-name-server 172.16.0.3;

# temps de bail
default-lease-time 600;
max-lease-time 7200;

# DMZ
subnet 172.16.14.0 netmask 255.255.255.0 {
    range 172.16.14.3 172.16.14.254;
    option routers 172.16.14.1;
    deny unknown-clients;
}

# local
subnet 172.16.24.0 netmask 255.255.255.0 {
    range 172.16.24.4 172.16.24.254;
    option routers 172.16.24.1;
    deny unknown-clients;
}

# host dmz-server
host dmz-server {
    hardware ethernet 00:05:5d:69:d1:30;
    fixed-adress 172.16.14.2;
}

# host local1
host loca1 {
    hardware ethernet 00:12:3f:62:a4:7b;
    fixed-adress 172.16.24.2;
}

# host local2
host dmz-server {
    hardware ethernet 00:12:3f:3b:f6:e5;
    fixed-adress 172.16.24.3;
}

# host bp-laptop
host bp-laptop {
    hardware ethernet xx:xx:xx:xx:xx:xx;
}
\end{verbatim}

Pour démarrer le serveur DHCP, exécuter la commande suivante~:
\begin{verbatim}
dhcpd eth1 eth2
\end{verbatim}

Afin de démarrer le démon avec l'ordinateur, ajouter cette même commande au
fichier \verb#/etc/rc.d/rc.local#.

\subsection{Configuration des clients}

\subsubsection{DMZ}
Modifier le fichier \verb#/etc/rc.d/rc.inet1.conf# sur \emph{dmz-server} comme
suit~:
\begin{verbatim}
# Config information for eth0:
IPADDR[0]=""
NETMASK[0]=""
USE_DHCP[0]="yes"
DHCP_HOSTNAME[0]="dmz-server"
\end{verbatim}

\subsubsection{Local}
Modifier le fichier \verb#/etc/rc.d/rc.inet1.conf# sur \emph{local1} comme
suit~:
\begin{verbatim}
# Config information for eth0:
IPADDR[0]=""
NETMASK[0]=""
USE_DHCP[0]="yes"
DHCP_HOSTNAME[0]="local1"
\end{verbatim}

Modifier le fichier \verb#/etc/rc.d/rc.inet1.conf# sur \emph{local2} comme
suit~:
\begin{verbatim}
# Config information for eth0:
IPADDR[0]=""
NETMASK[0]=""
USE_DHCP[0]="yes"
DHCP_HOSTNAME[0]="local2"
\end{verbatim}

Exécuter la commande \verb#dhcpcd# (ou \verb#dhclient#) pour lancer le démon du
client DHCP sur chacune des machines.

\section{Conclusion}
Toutes les machines de notre réseau, excepté la passerelle, recoivent une
adresse IP du serveur DHCP. Celles-ci recoivent également un masque, une
passerelle et une route par défaut, ainsi qu'une adresse de serveur DNS. Ceci
facilite grandement la gestion du réseau, car nous pouvons gérer l'adressage de
celui-ci depuis une seule machine uniquement. Un filtrage par adresse MAC a été
établi dans la configuration du DHCP.

\end{document}
