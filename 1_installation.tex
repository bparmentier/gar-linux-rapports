\documentclass[11pt,a4paper,oneside]{article}
\usepackage[utf8]{inputenc}
\usepackage[T1]{fontenc}
\usepackage{lmodern}
\usepackage[french]{babel}
\pagestyle{headings}
\author{Bruno Parmentier \and Olivier Marcotte}
\title{Gestion et administration des réseaux - Linux \\[1cm] \emph{Installation de
Slackware}}
\date{27 février 2014}
\begin{document}

\begin{titlepage}
\maketitle
\thispagestyle{empty}
\end{titlepage}

\section{Objectif}
Partitionnement de disques et installation de la distribution GNU/Linux
Slackware 14.0 sur quatre machines Dell.

\section{Mode opératoire}

\subsection{Préparation}
Au démarrage de l'ordinateur, à l'étape POST, appuyer sur F12 afin de
sélectionner le boot sur la carte réseau (vérifier que l'option est activée dans
le BIOS). L'ordinateur va rechercher un serveur PXE sur le réseau. La machine
à l'adresse \verb#172.16.0.147# contient une image ISO de Slackware 14.0 qui va
étre récupérée pour l'installation.

\subsection{Partitionnement}
Une fois Slackware démarré sur la machine, sélectionner le bon layout de clavier
(be-latin1 dans notre cas).
Taper \verb#fdisk -l# pour visionner les partitions existantes.

Le disque sur lequel nous allons faire l'installation est \verb#/dev/sda#. Nous
allons créer deux partitions primaires. La première, \verb#/dev/sda1# (10 Gio),
contiendra le root \verb#/#, la deuxième, \verb#/dev/sda2# (2 Gio), sera la
partition \verb#swap#.

\subsection{Installation}
Lancer le script d'installation avec la commande \verb#setup#.
\begin{enumerate}
    \item \verb#KEYMAP# : \verb#azerty/be-latin1.map#
    \item \verb#ADDSWAP# : \verb#/dev/sda2# OK
    \item \verb#TARGET# : \verb#/dev/sda1# OK $\rightarrow$ ext2
    \item \verb#SOURCE# : Install from NFS $\rightarrow$ DHCP $\rightarrow$ NFS
        server IP : \verb#172.16.0.147# $\rightarrow$ Slackware directory :
        \verb#/pub/slackware-14.00#
    \item \verb#SELECT# : laisser les options par défaut
    \item \verb#INSTALL# : full
\end{enumerate}

\section{Conclusion}
L'étape de partitionnement du disque est l'étape la plus délicate de
l'installation. Tout s'est cependant bien déroulé, la distribution est
installée et est fonctionnelle.
\end{document}
